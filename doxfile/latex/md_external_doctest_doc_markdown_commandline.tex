{\bfseries{doctest}} works quite nicely without any command line options at all -\/ but for more control a bunch are available.

{\bfseries{Query flags}} -\/ after the result is printed the program quits without executing any test cases (and if the framework is integrated into a client codebase which $\ast$$\ast$supplies it\textquotesingle{}s own \`{}\`{}\`{}main()\`{}\`{}\`{} entry point$\ast$$\ast$ -\/ the program should check the result of {\ttfamily should\+Exit()} method after calling {\ttfamily run()} on a {\ttfamily doctest\+::\+Context} object and should exit -\/ this is left up to the user).

{\bfseries{Int/\+String options}} -\/ they require a value after the {\ttfamily =} sign -\/ without spaces! For example\+: {\ttfamily -\/-\/order-\/by=rand}.

{\bfseries{Bool options}} -\/ they expect {\ttfamily 1}/{\ttfamily yes}/{\ttfamily on}/{\ttfamily true} or {\ttfamily 0}/{\ttfamily no}/{\ttfamily off}/{\ttfamily false} after the {\ttfamily =} sign -\/ but they can also be used like flags and the {\ttfamily =value} part can be skipped -\/ then {\ttfamily true} is assumed. ~\newline


{\bfseries{Filters}} use wildcards for matching values -\/ where {\ttfamily $\ast$} means \char`\"{}match any sequence\char`\"{} and {\ttfamily ?} means \char`\"{}match any one character\char`\"{}. To pass a pattern with an interval use {\ttfamily \char`\"{}\char`\"{}} like this\+: {\ttfamily -\/-\/test-\/case=\char`\"{}$\ast$no sound$\ast$,vaguely named test number ?\char`\"{}}.

All the options can also be set with code (defaults/overrides) if the user $\ast$$\ast$supplies the \`{}\`{}\`{}main()\`{}\`{}\`{} function$\ast$$\ast$.

\tabulinesep=1mm
\begin{longtabu}spread 0pt [c]{*{2}{|X[-1]}|}
\hline
\cellcolor{\tableheadbgcolor}\textbf{ Query Flags }&\PBS\centering \cellcolor{\tableheadbgcolor}\textbf{ Description  }\\\cline{1-2}
\endfirsthead
\hline
\endfoot
\hline
\cellcolor{\tableheadbgcolor}\textbf{ Query Flags }&\PBS\centering \cellcolor{\tableheadbgcolor}\textbf{ Description  }\\\cline{1-2}
\endhead
{\ttfamily -\/?} ~~~ {\ttfamily -\/-\/help} {\ttfamily -\/h} &Prints a help message listing all these flags/options  \\\cline{1-2}
{\ttfamily -\/v} ~~~ {\ttfamily -\/-\/version} &Prints the version of the {\bfseries{doctest}} framework  \\\cline{1-2}
{\ttfamily -\/c} ~~~ {\ttfamily -\/-\/count} &Prints the number of test cases matching the current filters (see below)  \\\cline{1-2}
{\ttfamily -\/ltc} {\ttfamily -\/-\/list-\/test-\/cases} &Lists all test cases by name which match the current filters (see below)  \\\cline{1-2}
{\ttfamily -\/lts} {\ttfamily -\/-\/list-\/test-\/suites} &Lists all test suites by name which have at least one test case matching the current filters (see below)  \\\cline{1-2}
{\ttfamily -\/lr} {\ttfamily -\/-\/list-\/reporters} &Lists all registered $\ast$$\ast$reporters$\ast$$\ast$  \\\cline{1-2}
{\bfseries{Int/\+String Options}} &\DoxyHorRuler{1}
  \\\cline{1-2}
{\ttfamily -\/tc} ~ {\ttfamily -\/-\/test-\/case=$<$filters$>$} &Filters test cases based on their name. By default all test cases match but if a value is given to this filter like {\ttfamily -\/-\/test-\/case=$\ast$math$\ast$,$\ast$sound$\ast$} then only test cases who match at least one of the patterns in the comma-\/separated list with wildcards will get executed/counted/listed  \\\cline{1-2}
{\ttfamily -\/tce} {\ttfamily -\/-\/test-\/case-\/exclude=$<$filters$>$} &Same as the {\ttfamily -\/test-\/case=$<$filters$>$} option but if any of the patterns in the comma-\/separated list of values matches -\/ then the test case is skipped  \\\cline{1-2}
{\ttfamily -\/sf} ~ {\ttfamily -\/-\/source-\/file=$<$filters$>$} &Same as {\ttfamily -\/-\/test-\/case=$<$filters$>$} but filters based on the file in which test cases are written  \\\cline{1-2}
{\ttfamily -\/sfe} {\ttfamily -\/-\/source-\/file-\/exclude=$<$filters$>$} &Same as {\ttfamily -\/-\/test-\/case-\/exclude=$<$filters$>$} but filters based on the file in which test cases are written  \\\cline{1-2}
{\ttfamily -\/ts} ~ {\ttfamily -\/-\/test-\/suite=$<$filters$>$} &Same as {\ttfamily -\/-\/test-\/case=$<$filters$>$} but filters based on the test suite in which test cases are in  \\\cline{1-2}
{\ttfamily -\/tse} {\ttfamily -\/-\/test-\/suite-\/exclude=$<$filters$>$} &Same as {\ttfamily -\/-\/test-\/case-\/exclude=$<$filters$>$} but filters based on the test suite in which test cases are in  \\\cline{1-2}
{\ttfamily -\/sc} ~ {\ttfamily -\/-\/subcase=$<$filters$>$} &Same as {\ttfamily -\/-\/test-\/case=$<$filters$>$} but filters subcases based on their names  \\\cline{1-2}
{\ttfamily -\/sce} {\ttfamily -\/-\/subcase-\/exclude=$<$filters$>$} &Same as {\ttfamily -\/-\/test-\/case-\/exclude=$<$filters$>$} but filters based on subcase names  \\\cline{1-2}
{\ttfamily -\/r} {\ttfamily -\/-\/reporters=$<$filters$>$} &List of $\ast$$\ast$reporters$\ast$$\ast$ to use (default is {\ttfamily console})  \\\cline{1-2}
{\ttfamily -\/o} ~ {\ttfamily -\/-\/out=$<$string$>$} &Output filename  \\\cline{1-2}
{\ttfamily -\/ob} ~ {\ttfamily -\/-\/order-\/by=$<$string$>$} &Test cases will be sorted before being executed either by {\bfseries{the file in which they are}} / {\bfseries{the test suite they are in}} / {\bfseries{their name}} / {\bfseries{random}}. The possible values of {\ttfamily $<$string$>$} are {\ttfamily file}/{\ttfamily suite}/{\ttfamily name}/{\ttfamily rand}. The default is {\ttfamily file}. {\bfseries{N\+O\+TE\+: the order produced by the {\ttfamily file}, {\ttfamily suite} and {\ttfamily name} options is compiler-\/dependent and might differ depending on the compiler used.}}  \\\cline{1-2}
{\ttfamily -\/rs} ~ {\ttfamily -\/-\/rand-\/seed=$<$int$>$} &The seed for random ordering  \\\cline{1-2}
{\ttfamily -\/f} ~~~ {\ttfamily -\/-\/first=$<$int$>$} &The {\bfseries{first}} test case to execute which passes the current filters -\/ for range-\/based execution -\/ see \href{../../examples/range_based_execution.py}{\texttt{ {\bfseries{the example python script}}}}  \\\cline{1-2}
{\ttfamily -\/l} ~~~ {\ttfamily -\/-\/last=$<$int$>$} &The {\bfseries{last}} test case to execute which passes the current filters -\/ for range-\/based execution -\/ see \href{../../examples/range_based_execution.py}{\texttt{ {\bfseries{the example python script}}}}  \\\cline{1-2}
{\ttfamily -\/aa} ~ {\ttfamily -\/-\/abort-\/after=$<$int$>$} &The testing framework will stop executing test cases/assertions after this many failed assertions. The default is 0 which means don\textquotesingle{}t stop at all. Note that the framework uses an exception to stop the current test case regardless of the level of the assert ({\ttfamily C\+H\+E\+CK}/{\ttfamily R\+E\+Q\+U\+I\+RE}) -\/ so be careful with asserts in destructors...  \\\cline{1-2}
{\ttfamily -\/scfl} {\ttfamily -\/-\/subcase-\/filter-\/levels=$<$int$>$} &Apply subcase filters only for the first {\ttfamily $<$int$>$} levels of nested subcases and just run the ones nested deeper. Default is a very high number which means {\itshape filter any subcase}  \\\cline{1-2}
{\bfseries{Bool Options}} &\DoxyHorRuler{1}
  \\\cline{1-2}
{\ttfamily -\/s} ~~~ {\ttfamily -\/-\/success=$<$bool$>$} &To include successful assertions in the output  \\\cline{1-2}
{\ttfamily -\/cs} ~ {\ttfamily -\/-\/case-\/sensitive=$<$bool$>$} &Filters being treated as case sensitive  \\\cline{1-2}
{\ttfamily -\/e} ~~~ {\ttfamily -\/-\/exit=$<$bool$>$} &Exits after the tests finish -\/ this is meaningful only when the client has $\ast$$\ast$provided the \`{}\`{}\`{}main()\`{}\`{}\`{} entry point$\ast$$\ast$ -\/ the program should check the {\ttfamily should\+Exit()} method after calling {\ttfamily run()} on a {\ttfamily doctest\+::\+Context} object and should exit -\/ this is left up to the user. The idea is to be able to execute just the tests in a client program and to not continue with it\textquotesingle{}s execution  \\\cline{1-2}
{\ttfamily -\/d} ~ {\ttfamily -\/-\/duration=$<$bool$>$} &Prints the time each test case took in seconds  \\\cline{1-2}
{\ttfamily -\/nt} ~ {\ttfamily -\/-\/no-\/throw=$<$bool$>$} &Skips \href{assertions.md\#exceptions}{\texttt{ {\bfseries{exceptions-\/related assertion}}}} checks  \\\cline{1-2}
{\ttfamily -\/ne} ~ {\ttfamily -\/-\/no-\/exitcode=$<$bool$>$} &Always returns a successful exit code -\/ even if a test case has failed  \\\cline{1-2}
{\ttfamily -\/nr} ~ {\ttfamily -\/-\/no-\/run=$<$bool$>$} &Skips all runtime {\bfseries{doctest}} operations (except the test registering which happens before the program enters {\ttfamily main()}). This is useful if the testing framework is integrated into a client codebase which has $\ast$$\ast$provided the \`{}\`{}\`{}main()\`{}\`{}\`{} entry point$\ast$$\ast$ and the user wants to skip running the tests and just use the program  \\\cline{1-2}
{\ttfamily -\/nv} ~ {\ttfamily -\/-\/no-\/version=$<$bool$>$} &Omits the framework version in the output  \\\cline{1-2}
{\ttfamily -\/nc} ~ {\ttfamily -\/-\/no-\/colors=$<$bool$>$} &Disables colors in the output  \\\cline{1-2}
{\ttfamily -\/fc} ~ {\ttfamily -\/-\/force-\/colors=$<$bool$>$} &Forces the use of colors even when a tty cannot be detected  \\\cline{1-2}
{\ttfamily -\/nb} ~ {\ttfamily -\/-\/no-\/breaks=$<$bool$>$} &Disables breakpoints in debuggers when an assertion fails  \\\cline{1-2}
{\ttfamily -\/ns} ~ {\ttfamily -\/-\/no-\/skip=$<$bool$>$} &Don\textquotesingle{}t skip test cases marked as skip with a decorator  \\\cline{1-2}
{\ttfamily -\/gfl} {\ttfamily -\/-\/gnu-\/file-\/line=$<$bool$>$} &{\ttfamily \+:n\+:} vs {\ttfamily (n)\+:} for line numbers in output (gnu mode is usually for linux tools/\+I\+D\+Es and is with the {\ttfamily \+:} separator)  \\\cline{1-2}
{\ttfamily -\/npf} {\ttfamily -\/-\/no-\/path-\/filenames=$<$bool$>$} &Paths are removed from the output when a filename is printed -\/ useful if you want the same output from the testing framework on different environments  \\\cline{1-2}
{\ttfamily -\/nln} {\ttfamily -\/-\/no-\/line-\/numbers=$<$bool$>$} &Line numbers are replaced with {\ttfamily 0} in the output when a source location is printed -\/ useful if you want the same output from the testing framework even when test positions change within a source file  \\\cline{1-2}
~~~~~~~~~~~~~~~~~~~~~~~~~~~~~~~~~~~~~~~~~~~~~~~~~~~~~~~~~~~~~~~~~ &\\\cline{1-2}
\end{longtabu}


All the flags/options also come with a prefixed version (with {\ttfamily -\/-\/dt-\/} at the front by default) -\/ for example {\ttfamily -\/-\/version} can be used also with {\ttfamily -\/-\/dt-\/version} or {\ttfamily -\/-\/dt-\/v}.

The default prefix is {\ttfamily -\/-\/dt-\/}, but this can be changed by setting the \href{configuration.md\#doctest_config_options_prefix}{\texttt{ $\ast$$\ast${\ttfamily D\+O\+C\+T\+E\+S\+T\+\_\+\+C\+O\+N\+F\+I\+G\+\_\+\+O\+P\+T\+I\+O\+N\+S\+\_\+\+P\+R\+E\+F\+IX}$\ast$$\ast$}} define.

All the unprefixed versions listed here can be disabled with the \href{configuration.md\#doctest_config_no_unprefixed_options}{\texttt{ $\ast$$\ast${\ttfamily D\+O\+C\+T\+E\+S\+T\+\_\+\+C\+O\+N\+F\+I\+G\+\_\+\+N\+O\+\_\+\+U\+N\+P\+R\+E\+F\+I\+X\+E\+D\+\_\+\+O\+P\+T\+I\+O\+NS}$\ast$$\ast$}} define.

This is done for easy interoperability with client command line option handling when the testing framework is integrated within a client codebase -\/ all {\bfseries{doctest}} related flags/options can be prefixed so there are no clashes and so that the user can exclude everything starting with {\ttfamily -\/-\/dt-\/} from their option parsing.

If there isn\textquotesingle{}t an option to exclude those starting with {\ttfamily -\/-\/dt-\/} then the {\ttfamily dt\+\_\+removed} helper class might help to filter them out\+:


\begin{DoxyCode}{0}
\DoxyCodeLine{ \{c++\}}
\DoxyCodeLine{\#define DOCTEST\_CONFIG\_NO\_UNPREFIXED\_OPTIONS}
\DoxyCodeLine{\#define DOCTEST\_CONFIG\_IMPLEMENT}
\DoxyCodeLine{\#include "doctest.h"}
\DoxyCodeLine{}
\DoxyCodeLine{class dt\_removed \{}
\DoxyCodeLine{    std::vector<const char*> vec;}
\DoxyCodeLine{public:}
\DoxyCodeLine{    dt\_removed(const char** argv\_in) \{}
\DoxyCodeLine{        for(; *argv\_in; ++argv\_in)}
\DoxyCodeLine{            if(strncmp(*argv\_in, "-\/-\/dt-\/", strlen("-\/-\/dt-\/")) != 0)}
\DoxyCodeLine{                vec.push\_back(*argv\_in);}
\DoxyCodeLine{        vec.push\_back(NULL);}
\DoxyCodeLine{    \}}
\DoxyCodeLine{}
\DoxyCodeLine{    int          argc() \{ return static\_cast<int>(vec.size()) -\/ 1; \}}
\DoxyCodeLine{    const char** argv() \{ return \&vec[0]; \} // Note: non-\/const char **:}
\DoxyCodeLine{\};}
\DoxyCodeLine{}
\DoxyCodeLine{int program(int argc, const char** argv);}
\DoxyCodeLine{}
\DoxyCodeLine{int main(int argc, const char** argv) \{}
\DoxyCodeLine{    doctest::Context context(argc, argv);}
\DoxyCodeLine{    int test\_result = context.run(); // run queries, or run tests unless -\/-\/no-\/run}
\DoxyCodeLine{}
\DoxyCodeLine{    if(context.shouldExit()) // honor query flags and -\/-\/exit}
\DoxyCodeLine{        return test\_result;}
\DoxyCodeLine{}
\DoxyCodeLine{    dt\_removed args(argv);}
\DoxyCodeLine{    int app\_result = program(args.argc(), args.argv());}
\DoxyCodeLine{}
\DoxyCodeLine{    return test\_result + app\_result; // combine the 2 results}
\DoxyCodeLine{\}}
\DoxyCodeLine{}
\DoxyCodeLine{int program(int argc, const char** argv) \{}
\DoxyCodeLine{    printf("Program: \%d arguments received:\(\backslash\)n", argc -\/ 1);}
\DoxyCodeLine{    while(*++argv)}
\DoxyCodeLine{        printf("'\%s'\(\backslash\)n", *argv);}
\DoxyCodeLine{    return EXIT\_SUCCESS;}
\DoxyCodeLine{\}}
\end{DoxyCode}


When ran like this\+:


\begin{DoxyCode}{0}
\DoxyCodeLine{program.exe -\/-\/dt-\/test-\/case=math* -\/-\/my-\/option -\/s -\/-\/dt-\/no-\/breaks}
\end{DoxyCode}


Will output this\+:


\begin{DoxyCode}{0}
\DoxyCodeLine{Program: 2 arguments received:}
\DoxyCodeLine{'-\/-\/my-\/option'}
\DoxyCodeLine{'-\/s'}
\end{DoxyCode}


\DoxyHorRuler{0}


\href{readme.md\#reference}{\texttt{ Home}}



